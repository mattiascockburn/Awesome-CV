\newpage
%-------------------------------------------------------------------------------
%	SECTION TITLE
%-------------------------------------------------------------------------------
\cvsection{Charlas / Conferencias}


%-------------------------------------------------------------------------------
%	CONTENT
%-------------------------------------------------------------------------------
\begin{cventries}

%---------------------------------------------------------

  \cventry
    {Presentador: Administración de sistemas con Puppet y Foreman}
    {ContinuousLifecycle 2016}
    {Mannheim, Alemania}
    {Septiembre 2016}
    {
      \begin{cvitems}
        \item {Cómo se puede usar Foreman y Puppet para proporcionar una forma completamente automática para implementar y configurar sistemas en entornos altamente complejos}
      \end{cvitems}
    }
  \cventry
    {Presentador: Provisión de aplicaciones de múltiples niveles con Docker}
    {Committerconf 2015}
    {Essen, Alemania}
    {Noviembre 2015}
    {
      \begin{cvitems}
      \item {Introducción de docker-machine, docker-compose y docker-swarm y cómo se pueden usar para implementar aplicaciones de varios niveles en un cluster de docker-swarm privado}
      \end{cvitems}
    }
  \cventry
    {Presentador: Implementación de una solución de provisión basada en SUSE usando SUSE Manager en la óptica Apollo (cadena óptica de Europa)}
    {SUSECon 2015}
    {Amsterdam, Holanda}
    {Noviembre 2015}
    {
      \begin{cvitems}
      \item {Presentación de la solución desarrollada por B1 utilizada para automatizar todo el proceso de provisión de servidores utilizados por la óptica Apollo a nuevas sucursales (gran cadena minorista europea de gafas y accesorios relacionados)}
      \end{cvitems}
    }
  \cventry
    {Presentador: Intregración continua con Docker y Jenkins}
    {LinuxCon Europe 2014} % Event
    {Düsseldorf, Alemania} % Location
    {Octubre 2014} % Date(s)
    {
      \begin{cvitems} % Description(s)
      \item {Introducción de tecnologías / técnicas que pueden ser usadas para contruir un CI/CD pipeline basado en Docker y Jenkins}
      \end{cvitems}
    }
  \cventry
    {Presentador: Creación de una solución dinámica de provisión de software utilizando software libre}
    {LinuxTag 2014} % Event
    {Berlin, Alemania} % Location
    {Mayo 2014} % Date(s)
    {
      \begin{cvitems} % Description(s)
      \item {Cómo componentes de software (como Gitlab, Open Build Service, Jenkins, Spacewalk y Puppet / Ansible) disponibles libremente pueden ser utilizados para crear un pipeline de software delivery adecuado.}
      \end{cvitems}
    }

%---------------------------------------------------------
\end{cventries}
