%-------------------------------------------------------------------------------
%	SECTION TITLE
%-------------------------------------------------------------------------------
\cvsection{Work Experience}


%-------------------------------------------------------------------------------
%	CONTENT
%-------------------------------------------------------------------------------
\cvsubsection{Key projects}
\begin{cventries}
  \cventry
    {Retail chain}
    {Implementation of a streamlined automatic deployment solution for RHEL/CentOS 7 and Windows
      Server 2012 R2/2016}
    {}
    {}
    {
      \begin{cvitems}
      \item Creation of a completely new deployment platform based on the autoinstallation
        frameworks of Windows Server 2012 R2/2016 (unattend.xml/Windows PE) and RedHat Enterprise
        Linux/CentOS (Kickstart) based on Katello/Foreman for completely unattend provisioning.
      \item Windows Deployment was done using a custom Windows PE image, bypassing the regular
        installation mechanism to better automate the deployment process and for tighter integration
        into Foreman and its feature set.
      \item Development of a Foreman smart-proxy plugin (Ruby) in order to orchestrate DHCP/DNS infrastructure (Bluecat DHCP/DNS Server/IPAM)
      \item Seemless installation of new bare metal as well as virtual machines hosted on VMWare vSphere.
      \item Automatic configuration of both
        platforms with the help of Puppet 4.x/5.x, including automatic Active Directory Joins and
        basic application deployment (chocolatey on Windows). Automatic integration with patch management
        systems (Katello for RHEL/CentOS) and WSUS for Windows Server platforms.
      \item Infrastructure landscape: several hundred to thousand machines with a multitude of
        different networks and DMZs
      \item Technologies used: Puppet 4.x/5.x, Foreman 1.13-1.15, Katello 3.x, Ruby 2.1, Kickstart, Windows PE scripting, Windows
        MDT, chocolatey, RHEL/CentOS 7.x, Windows Server 2012 R2/2016, Bluecat IPAM,DHCP/DNS Server,
        bash
      \end{cvitems}
    }

  \cventry
    {Financial services provider} {Implementation of an automated patch process for RHEL 5-7 machines with complete audit/backlog} {}
    {}
    {
      \begin{cvitems}
      \item Being bound to security standards like PCI-DSS my company was approached by a german
        fincial service provider to help automate their patch procedures.
      \item Several tools had to be
        integrated/orchestrated: RedHat Satellite 5.x (Patch management tool), Atlassian Jira
        (ticketing system/audit), National Vulnerability Database (nvd.nist.gov).
      \item Creation of a bridge to nvd.nist.gov in order to better classify security patches from
        RedHat against the CVSS base score of the NVD because of PCI-DSS requirements
      \item Development of a bridge from Satellite into the ticketing system. The bridge
        automatically opens new tickets with specified subtasks (as demanded by the customer). The
        bridge automatically keeps track of the patching process, updates tickets as systems are
        getting patched and automatically closes each specific ticket as soon as patching a specific
        vulnerability is completed.
      \item Additionally an easy to use text-mode frontend had to be created for the systems engineer to initiate the patching itself.
      \item Upon conpletion of this project the overall time needed to manage the monthly patching
        process was reduced from roughly 2-3 days to around 5 minutes.
      \item Technologies used: RedHat Satellite 5.x and its APIs, Python 2.7, Atlassian Jira APIs,
        nvd.nist.gov XML-dumps, bash
      \end{cvitems}
    }

  \cventry
    {Provider of technical trainings}
    {Implementation of an automatic deployment system for Windows 7 and technical training resources based on Linux/Bittorrent}
    {}
    {}
    {
      \begin{cvitems}
      \item The training provider i worked at needed a better process to deploy the training materials/virtual machines and configuration needed for new courses. As there are different courses with their own requirements each week, every workstation needed to be deployed at least once a week.
      \item To work around the limitations of Windows and its deployment process a new (at the time) feature of Windows was used: Direct booting of WMI-Files
      \item With the help of a custom Linux live system booted via PXE, a new Windows Image will deployed to the machine
      \item The live system obtains a list of needed resources for the next course from a central application (developed in PHP)
      \item Then the relevant courseware (with peer2peer download via bittorrent) is copied to the local machine. The usage of bittorrent is critical here because several Microsoft technical courses are huge in terms of filesize (VM Images, Setups, ISO Files)
      \item After the first boot, several VBScript and Powershell based scripts configure the
        machine in order to support the next weeks course. Group policies handle the generic
        configuration of the machines.
      \item Technologies used: Debian preseed/liveiso tools, bash, aria2c/bittorrent, Windows WMI
        Image format, VBS, Powershell, Active Directory Group Policies, PHP
      \end{cvitems}
    }
\end{cventries}

\cvsubsection{Work history}
\begin{cventries}

%---------------------------------------------------------
  \cventry
    {Solution Architect} % Job title
    {B1 Systems GmbH} % Organization
    {Vohburg, Germany} % Location
    {May 2011 - PRESENT} % Date(s)
    {
      \begin{cvitems} % Description(s) of tasks/responsibilities
        \item Delivering solutions for infrastructure automation and ease of administration in a variety of companies ranging from banks/insurance companies to automobile manufacturers and digital media agencies
        \item Conducting trainings in the field of configuration management and systems management technologies
        \item Giving talks about those topics at various conferences as well as writing articles in highly respected german subject literature
        \item Typical technologies used for projects include RedHat Enterprise Linux 5-7 (including Kickstart), SUSE Linux Enterprise 11+ (with AutoYaST), Debian/Ubuntu Linux (with preseed), Puppet (with PuppetDB, hiera, r10k, rspec-puppet, beaker/testkitchen), Ansible (with ansible-vault, dynamic inventories), Chef, Salt, bash, Docker (Machine, Compose, Swarm), Python 2/3, Ruby, RedHat Satellite/Spacewalk/SUSE Manager, vagrant, packer, libvirt/KVM/Xen/VMWare
      \end{cvitems}
    }
%    \cvsubentry
%    {Key projects}
%    {}
%    {}
%    {}
%    {
%      \begin{cvitems}
%      \item foo
%      \item bar
%      \end{cvitems}
%    }

%---------------------------------------------------------
  \cventry
    {System Administrator \& Teacher} % Job title
    {signet GmbH} % Organization
    {Kassel, Germany} % Location
    {Oct. 2005 - Apr. 2011} % Date(s)
    {
      \begin{cvitems} % Description(s) of tasks/responsibilities
        \item foo
      \end{cvitems}
    }

%---------------------------------------------------------
\end{cventries}
